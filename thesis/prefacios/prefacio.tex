\chapter*{}
%\thispagestyle{empty}
%\cleardoublepage

%\thispagestyle{empty}

\begin{titlepage}


    \setlength{\centeroffset}{-0.5\oddsidemargin}
    \addtolength{\centeroffset}{0.5\evensidemargin}
    \thispagestyle{empty}

    \noindent\hspace*{\centeroffset}\begin{minipage}{\textwidth}

        \centering
        %\includegraphics[width=0.9\textwidth]{imagenes/logo_ugr.jpg}\\[1.4cm]

        %\textsc{ \Large PROYECTO FIN DE CARRERA\\[0.2cm]}
        %\textsc{ INGENIERÍA EN INFORMÁTICA}\\[1cm]
        % Upper part of the page
        % 

        \vspace{3.3cm}

        %si el proyecto tiene logo poner aquí
        \includegraphics[width=0.9\textwidth]{imagenes/logo_ugr.jpg}
        \vspace{0.5cm}

        % Title

        {\Huge\bfseries DockEEG\\
        }
        \noindent\rule[-1ex]{\textwidth}{3pt}\\[3.5ex]
        {\large\bfseries Dockerización de Aplicación Paralela y Distribuida para Clasificación de EEGs: Análisis de Viabilidad y Rendimiento\\[4cm]}
    \end{minipage}

    \vspace{2.5cm}
    \noindent\hspace*{\centeroffset}\begin{minipage}{\textwidth}
        \centering

        \textbf{Autor}\\ {Fernando Cuesta Bueno}\\[2.5ex]
        \textbf{Tutor}\\
        {Juan José Escobar Pérez}\\[2cm]
        %\includegraphics[width=0.15\textwidth]{imagenes/tstc.png}\\[0.1cm]
        %\textsc{Departamento de Teoría de la Señal, Telemática y Comunicaciones}\\
        %\textsc{---}\\
        %Granada, mes de 201
    \end{minipage}
    %\addtolength{\textwidth}{\centeroffset}
    \vspace{\stretch{2}}


\end{titlepage}






\cleardoublepage
\thispagestyle{empty}

\begin{center}
       {\large\bfseries DockEEG - Dockerización de Aplicación Paralela y Distribuida para Clasificación de EEGs: Análisis de Viabilidad y Rendimiento}\\
\end{center}
\begin{center}
       Fernando Cuesta Bueno\\
\end{center}

%\vspace{0.7cm}
\noindent{\textbf{Palabras clave}: contenerización, computación paralela, computación distribuida, EEG, Docker, Podman, rendimiento, escalabilidad, portabilidad, GPU, HPC}\\

\vspace{0.7cm}
\noindent{\textbf{Resumen}}\\

Este trabajo presenta un estudio exhaustivo sobre la viabilidad y el rendimiento de la contenerización de aplicaciones paralelas y distribuidas en el ámbito del procesamiento de señales EEG. Se ha desarrollado y evaluado DockEEG, una solución basada en contenedores (Docker y Podman) para la ejecución eficiente y portable de la aplicación HPMoon en distintos sistemas operativos (Ubuntu, Windows, Mac) y arquitecturas (CPU y GPU).

A lo largo del trabajo se han realizado experimentos sistemáticos para analizar el impacto de la contenerización en el rendimiento, la escalabilidad y la portabilidad, comparando la ejecución nativa y contenerizada en escenarios mononodo y multinodo. Los resultados muestran que el uso de contenedores no introduce penalizaciones relevantes, permitiendo mantener un rendimiento muy similar al de la ejecución nativa y facilitando la reproducibilidad y despliegue en entornos heterogéneos.

El estudio identifica el punto óptimo de paralelismo en torno a 8 hebras por nodo y destaca la importancia de ajustar el número de nodos para evitar sobrecargas de coordinación. Además, se analiza el impacto de la aceleración por GPU, que ofrece mejoras significativas en mononodo, aunque su escalabilidad es limitada en entornos distribuidos.

Finalmente, se proponen líneas de trabajo futuro orientadas a mejorar la escalabilidad multinodo, la gestión avanzada de recursos heterogéneos, la automatización del ciclo experimental, la validación en clústeres HPC y la explotación combinada de CPU y GPU en sistemas Mac. HPMoon demuestra así ser una herramienta eficaz y portable para la investigación y desarrollo de aplicaciones científicas

\cleardoublepage


\thispagestyle{empty}


\begin{center}
       {\large\bfseries DockEEG - Dockerization of a Parallel and Distributed Application for EEG Classification: Feasibility and Performance Analysis}\\
\end{center}
\begin{center}
       Fernando Cuesta Bueno\\
\end{center}

%\vspace{0.7cm}
\noindent{\textbf{Keywords}: containerization, parallel computing, distributed computing, EEG, Docker, Podman, performance, scalability, portability, GPU, HPC}\\

\vspace{0.7cm}
\noindent{\textbf{Abstract}}\\

This work presents a comprehensive study on the feasibility and performance of containerizing parallel and distributed applications in the field of EEG signal processing. DockEEG, a container-based solution (using Docker and Podman), has been developed and evaluated for the efficient and portable execution of the HPMoon application across different operating systems (Ubuntu, Windows, Mac) and architectures (CPU and GPU).

Throughout this work, systematic experiments have been conducted to analyze the impact of containerization on performance, scalability, and portability, comparing native and containerized execution in both single-node and multi-node scenarios. The results show that the use of containers does not introduce significant overhead, allowing performance very similar to native execution and facilitating reproducibility and deployment in heterogeneous environments.

The study identifies the optimal level of parallelism at around 8 threads per node and highlights the importance of adjusting the number of nodes to avoid coordination overhead. Additionally, the impact of GPU acceleration is analyzed, showing significant improvements in single-node setups, although scalability is limited in distributed environments.

Finally, future work is proposed to improve multi-node scalability, advanced management of heterogeneous resources, automation of the experimental cycle, validation in HPC clusters, and combined exploitation of CPU and GPU on Mac systems. HPMoon thus proves to be an effective and portable tool for scientific application research and development.

\chapter*{}
\thispagestyle{empty}

\noindent\rule[-1ex]{\textwidth}{2pt}\\[4.5ex]

Yo, \textbf{Fernando Cuesta Bueno}, alumno de la titulación Graduado en Ingeniería Informática de la \textbf{Escuela Técnica Superior
       de Ingenierías Informática y de Telecomunicación de la Universidad de Granada}, con DNI 77150866B, autorizo la
ubicación de la siguiente copia de mi Trabajo Fin de Grado en la biblioteca del centro para que pueda ser
consultada por las personas que lo deseen.

\vspace{6cm}

\noindent Fdo: Fernando Cuesta Bueno

\vspace{2cm}

\begin{flushright}
       Granada a 5 de septiembre de 2025.
\end{flushright}


\chapter*{}
\thispagestyle{empty}

\noindent\rule[-1ex]{\textwidth}{2pt}\\[4.5ex]

D. \textbf{Juan José Escobar Pérez}, Profesor del Departamento de Lenguajes y Sistemas Informáticos de la Universidad de Granada.

\vspace{0.5cm}

\textbf{Informa:}

\vspace{0.5cm}

Que el presente trabajo, titulado \textit{\textbf{Dockerización de Aplicación Paralela y Distribuida para Clasificación de EEGs: Análisis de Viabilidad y Rendimiento, DockEEG}},
ha sido realizado bajo su supervisión por \textbf{Fernando Cuesta Bueno}, y autorizo la defensa de dicho trabajo ante el tribunal
que corresponda.

\vspace{0.5cm}

Y para que conste, expiden y firman el presente informe en Granada a 5 de septiembre de 2025.

\vspace{1cm}

\textbf{El tutor:}

\vspace{5cm}

\noindent \textbf{Juan José Escobar Pérez}

\chapter*{Agradecimientos}
\thispagestyle{empty}

\vspace{1cm}


A mi madre Mercedes y a mi hermana Marta, todo lo que soy es gracias a vosotras. Todo lo que seré será por vosotras. Gracias por vuestro apoyo incondicional y por estar siempre ahí.

