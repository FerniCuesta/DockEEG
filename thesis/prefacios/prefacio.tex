\chapter*{}
%\thispagestyle{empty}
%\cleardoublepage

%\thispagestyle{empty}

%\begin{titlepage}


    \setlength{\centeroffset}{-0.5\oddsidemargin}
    \addtolength{\centeroffset}{0.5\evensidemargin}
    \thispagestyle{empty}

    \noindent\hspace*{\centeroffset}\begin{minipage}{\textwidth}

        \centering
        %\includegraphics[width=0.9\textwidth]{imagenes/logo_ugr.jpg}\\[1.4cm]

        %\textsc{ \Large PROYECTO FIN DE CARRERA\\[0.2cm]}
        %\textsc{ INGENIERÍA EN INFORMÁTICA}\\[1cm]
        % Upper part of the page
        % 

        \vspace{3.3cm}

        %si el proyecto tiene logo poner aquí
        \includegraphics[width=0.9\textwidth]{imagenes/logo_ugr.jpg}
        \vspace{0.5cm}

        % Title

        {\Huge\bfseries DockEEG\\
        }
        \noindent\rule[-1ex]{\textwidth}{3pt}\\[3.5ex]
        {\large\bfseries Dockerización de Aplicación Paralela y Distribuida para Clasificación de EEGs: Análisis de Viabilidad y Rendimiento\\[4cm]}
    \end{minipage}

    \vspace{2.5cm}
    \noindent\hspace*{\centeroffset}\begin{minipage}{\textwidth}
        \centering

        \textbf{Autor}\\ {Fernando Cuesta Bueno}\\[2.5ex]
        \textbf{Tutor}\\
        {Juan José Escobar Pérez}\\[2cm]
        %\includegraphics[width=0.15\textwidth]{imagenes/tstc.png}\\[0.1cm]
        %\textsc{Departamento de Teoría de la Señal, Telemática y Comunicaciones}\\
        %\textsc{---}\\
        %Granada, mes de 201
    \end{minipage}
    %\addtolength{\textwidth}{\centeroffset}
    \vspace{\stretch{2}}


\end{titlepage}






\cleardoublepage
\thispagestyle{empty}

%\vspace{0.7cm}
\noindent{\textbf{Palabras clave}: Contenerización, Computación de Altas Prestaciones, Tarjeta Gráfica, Electroencefalograma, Rendimiento, Escalabilidad, Portabilidad}\\

\vspace{0.7cm}
\noindent{\textbf{Resumen}}\\

Este trabajo presenta un estudio exhaustivo sobre la viabilidad y el rendimiento de la contenerización de aplicaciones paralelas y distribuidas en el ámbito del procesamiento de señales EEG. Para ello, se ha desarrollado y evaluado DockEEG, una solución basada en contenedores (Docker y Podman) que permite la ejecución eficiente y portable en distintos sistemas operativos (Ubuntu, Windows, MacOS) y arquitecturas (CPU y GPU).

A lo largo del trabajo se han realizado experimentos sistemáticos para analizar el impacto de la contenerización en el rendimiento, la escalabilidad y la portabilidad. Se ha comparado la ejecución nativa y contenerizada en escenarios multihebra y multiproceso, evaluando así las diferencias y ventajas de cada enfoque.

El estudio identifica el punto óptimo de paralelismo en torno a 8 hebras por proceso y resalta la importancia de ajustar el número de procesos para evitar sobrecargas de coordinación. Asimismo, se analiza el impacto de la aceleración por GPU, que ofrece mejoras significativas en entornos multihebra, aunque su escalabilidad resulta limitada en escenarios distribuidos.

Los resultados muestran que el uso de contenedores no introduce penalizaciones relevantes, permitiendo mantener un rendimiento muy similar al de la ejecución nativa y facilitando la reproducibilidad y el despliegue en entornos heterogéneos. Finalmente, se proponen líneas de trabajo futuro orientadas a mejorar la escalabilidad multinodo, la gestión avanzada de recursos heterogéneos, la automatización del ciclo experimental, la validación en clústeres HPC y la explotación combinada de CPU y GPU en sistemas Mac. DockEEG, demuestra así que los contenedores son una herramienta eficaz y portable para la investigación y desarrollo de aplicaciones científicas.

\cleardoublepage


\thispagestyle{empty}

%\vspace{0.7cm}
\noindent{\textbf{Keywords}: Containerization, High-Performance Computing, Graphics Processing Unit, Electroencephalogram, Performance, Scalability, Portability}\\

\vspace{0.7cm}
\noindent{\textbf{Abstract}}\\

This work presents a comprehensive study on the feasibility and performance of containerizing parallel and distributed applications in the field of EEG signal processing. To this end, DockEEG has been developed and evaluated: a container-based solution (Docker and Podman) that enables efficient and portable execution across different operating systems and architectures (CPU and GPU).

Throughout this work, systematic experiments have been carried out to analyze the impact of containerization on performance, scalability, and portability. Native and containerized execution have been compared in multithreaded and multiprocess scenarios, thus assessing the differences and advantages of each approach.

The study identifies the optimal point of parallelism around 8 threads per process and highlights the importance of tuning the number of processes to avoid coordination overhead. Likewise, the impact of GPU acceleration is analyzed, showing significant improvements in multithreaded environments, although its scalability proves limited in distributed scenarios.

The results show that the use of containers does not introduce relevant overhead, allowing performance to remain very close to that of native execution while facilitating reproducibility and deployment in heterogeneous environments. Finally, future work directions are proposed, aimed at improving multinode scalability, advanced management of heterogeneous resources, automation of the experimental cycle, validation on HPC clusters, and combined exploitation of CPU and GPU in Mac systems. This demonstrates that containers are an effective and portable tool for research and development of scientific applications.

\chapter*{}
\thispagestyle{empty}

\noindent\rule[-1ex]{\textwidth}{2pt}\\[4.5ex]

Yo, \textbf{Fernando Cuesta Bueno}, alumno de la titulación Graduado en Ingeniería Informática de la \textbf{Escuela Técnica Superior
       de Ingenierías Informática y de Telecomunicación de la Universidad de Granada}, con DNI 77150866B, autorizo la
ubicación de la siguiente copia de mi Trabajo Fin de Grado en la biblioteca del centro para que pueda ser
consultada por las personas que lo deseen.

\vspace{6cm}

\noindent Fdo: Fernando Cuesta Bueno

\vspace{2cm}

\begin{flushright}
       Granada a 5 de septiembre de 2025.
\end{flushright}


% \chapter*{}
% \thispagestyle{empty}

% \noindent\rule[-1ex]{\textwidth}{2pt}\\[4.5ex]

% D. \textbf{Juan José Escobar Pérez}, Profesor del Departamento de Lenguajes y Sistemas Informáticos de la Universidad de Granada.

% \vspace{0.5cm}

% \textbf{Informa:}

% \vspace{0.5cm}

% Que el presente trabajo, titulado \textit{\textbf{Dockerización de Aplicación Paralela y Distribuida para Clasificación de EEGs: Análisis de Viabilidad y Rendimiento}},
% ha sido realizado bajo su supervisión por \textbf{Fernando Cuesta Bueno}, y autorizo la defensa de dicho trabajo ante el tribunal
% que corresponda.

% \vspace{0.5cm}

% Y para que conste, expiden y firman el presente informe en Granada a 5 de septiembre de 2025.

% \vspace{1cm}

% \textbf{El tutor:}

% \vspace{5cm}

% \noindent \textbf{Juan José Escobar Pérez}

\chapter*{Agradecimientos}
\thispagestyle{empty}

\vspace{1cm}


A mi madre Mercedes y a mi hermana Marta, todo lo que soy es gracias a vosotras. Todo lo que seré será por vosotras. Gracias por vuestro apoyo incondicional y por estar siempre ahí.

