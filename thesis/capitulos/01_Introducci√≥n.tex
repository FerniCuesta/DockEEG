\chapter{Introducción}\label{cap:introduccion}

% [La introducción tiene que poner en contexto al lector contando, a modo de historia, el origen y contexto del problema, motivando por qué es necesario abordarlo y finalizando con lo que se propone en el proyecto.]

% [También es importante que sigas cierto orden y estructura a la hora de presentar (introducir) los contenidos, siguiendo un patrón que atienda al qué, para describir el contexto; al por qué, para dar razón o motivar el trabajo, y al por tanto, para definir objetivos consecuentes con la motivación y el contexto del trabajo.]

En el ámbito biomédico y de la bioingeniería, el análisis de señales de electroencefalograma (\ac{EEG}) para el desarrollo de Interfaces Cerebro-Computadora (Brain-Computer Interfaces, \ac{BCI}) constituye un reto de gran relevancia. Las \ac{BCI} permiten a personas con discapacidades motoras recuperar capacidades de comunicación y control sobre dispositivos externos, facilitando su autonomía y mejorando su calidad de vida. Asimismo, estas tecnologías se están explorando para aplicaciones médicas avanzadas, como la rehabilitación neurológica, la detección temprana de trastornos cerebrales y el control de prótesis inteligentes, lo que convierte el procesamiento eficiente de datos \ac{EEG} en un requisito crítico para traducir los avances científicos en beneficios tangibles para la sociedad \cite{mcfarland2017eeg, Lotte2015ElectroencephalographyB}.

La gran cantidad de datos generados por los \ac{EEG} y la necesidad de procesarlos en tiempo real requieren infraestructuras de cómputo altamente eficientes. En este contexto, la Computación de Alto Rendimiento (High Performance Computing, HPC) constituye un pilar fundamental. Gracias a su capacidad para ejecutar cálculos masivos en tiempos reducidos, HPC ha permitido abordar problemas antes inabordables, desde la simulación de fenómenos climáticos hasta el entrenamiento de modelos de inteligencia artificial de gran escala, y en particular el procesamiento de datos \ac{EEG} para \ac{BCI} \cite{2025HighPerformanceCF}.

Sin embargo, la utilización de HPC para procesar datos \ac{EEG} no está exenta de desafíos. El análisis de señales cerebrales de gran volumen y alta complejidad requiere paralelismo multinivel, arquitecturas heterogéneas y escalabilidad, además de una cuidadosa optimización de recursos para mantener la precisión en la clasificación mientras se minimizan los tiempos de procesamiento. Trabajos recientes han demostrado que la combinación de técnicas de reducción de dimensionalidad con infraestructuras HPC puede acelerar el procesamiento entre 1.5 y 4 veces sin sacrificar la precisión \cite{mcfarland2017eeg, Lotte2015ElectroencephalographyB, 2025HighPerformanceCF}, lo que evidencia la necesidad de entornos HPC eficientes y configurados adecuadamente.

Esta complejidad en la gestión de recursos y entornos de ejecución se ve agravada por dos problemas adicionales que limitan la eficacia de la ciencia computacional moderna: la creciente heterogeneidad de las arquitecturas y la falta de portabilidad y reproducibilidad de las aplicaciones. En numerosos casos, un software científico que funciona correctamente en un sistema falla en otro debido a diferencias en el hardware (CPU, GPU) o en las configuraciones de software (versiones de librerías, compiladores, dependencias del sistema). Esta situación no solo dificulta la verificación de resultados, sino que también limita la colaboración y el avance científico a gran escala.

En este contexto, la tecnología de contenedores ha emergido como una solución prometedora. Los contenedores permiten encapsular aplicaciones junto con todas sus dependencias en unidades portables y aisladas. De este modo, se garantiza que el software se ejecute de manera consistente en cualquier entorno, reduciendo la complejidad de despliegue y mejorando la reproducibilidad. Frente a las máquinas virtuales tradicionales, los contenedores presentan una sobrecarga mínima y ofrecen un rendimiento cercano al nativo, lo que los convierte en una alternativa atractiva para entornos HPC. En particular, \textit{Docker} se reconoce como la tecnología líder en contenerización debido a su baja sobrecarga, flexibilidad, portabilidad y capacidad de garantizar reproducibilidad \cite{saha2016evaluation}.

No obstante, diversos estudios han puesto de manifiesto que, en escenarios de computación de alto rendimiento, Docker todavía presenta ciertas limitaciones relacionadas con la seguridad y la latencia de red \cite{Alles2018AssessingTC, MedranoJaimes2018UseOC}. Estas restricciones abren un espacio de investigación particularmente relevante: evaluar hasta qué punto los contenedores pueden ser adoptados en entornos HPC sin comprometer el rendimiento ni la eficiencia, y bajo qué condiciones pueden convertirse en un elemento clave para mejorar la portabilidad y la reproducibilidad científica en sistemas heterogéneos.

Para poner a prueba esta propuesta, se ha seleccionado como caso de estudio el software \textit{HPMoon}, desarrollado en el marco de una tesis doctoral para la clasificación no supervisada de señales \ac{EEG}. \textit{HPMoon} constituye un escenario experimental idóneo al incorporar múltiples niveles de paralelismo (MPI, OpenMP, OpenCL), estar diseñado para arquitecturas heterogéneas con CPU y GPU, y contar con una base científica consolidada. Evaluar su ejecución tanto en entornos nativos como contenerizados permite analizar de manera rigurosa las ventajas, limitaciones y retos que supone la contenerización en aplicaciones HPC reales.

Este trabajo se enmarca, por tanto, en el estudio de la viabilidad y el impacto del uso de contenedores en aplicaciones científicas de alto rendimiento. La investigación realizada pretende contribuir a la mejora de la portabilidad, reproducibilidad y adopción de este tipo de aplicaciones en la comunidad científica, sin comprometer su rendimiento en arquitecturas modernas y heterogéneas.

\section{Motivación}\label{sec:motivacion}

% [Opcional si se ha motivado la realización del proyecto en los párrafos anteriores.]

La motivación principal de este trabajo surge de la necesidad de conciliar dos objetivos que, en ocasiones, parecen contrapuestos en la investigación científica computacional: por un lado, maximizar el rendimiento mediante arquitecturas HPC cada vez más complejas y heterogéneas, y por otro, garantizar la portabilidad y reproducibilidad de las aplicaciones en entornos diversos.

La contenerización representa una oportunidad única para cerrar esta brecha. Su adopción en entornos de HPC aún no es generalizada, en parte debido a la percepción de que puede introducir sobrecargas o limitar el acceso eficiente a los recursos de hardware, especialmente en configuraciones multinodo con GPU. Este trabajo busca arrojar luz sobre esta problemática mediante un análisis experimental detallado, aplicando la contenerización a un caso de uso real y exigente como \textit{HPMoon}.

En este sentido, el proyecto no se centra únicamente en medir tiempos de ejecución, sino en evaluar de manera integral la escalabilidad, la eficiencia en arquitecturas heterogéneas y la capacidad de reproducir resultados en múltiples plataformas. Con ello, se pretende ofrecer una contribución práctica y útil para la comunidad científica, ayudando a sentar las bases para una adopción más amplia y fundamentada de los contenedores en HPC.

De este modo, los apartados siguientes se centran en definir los objetivos concretos que guían esta investigación, tanto generales como específicos, y que permitirán estructurar el análisis y validar las hipótesis planteadas.

\section{Objetivos}\label{sec:objetivos}

Analizar la viabilidad y las limitaciones del uso de contenedores, concretamente Docker, para encapsular y ejecutar aplicaciones de alto rendimiento (HPC) en arquitecturas heterogéneas modernas —como big.LITTLE— y entornos multiplataforma, con el fin de facilitar su portabilidad, uso y adopción por parte de la comunidad científica.

\subsection{Objetivos específicos}\label{subsec:objetivos_especificos}

\begin{itemize}
   \item \textbf{OB1.} Investigar el estado del arte en el ámbito de la tecnología de contenedores y su aplicación en entornos de computación de alto rendimiento.
   \item \textbf{OB2.} Diseñar e implementar un conjunto de experimentos para evaluar el rendimiento de aplicaciones HPC contenerizadas en diferentes arquitecturas y plataformas.
   \item \textbf{OB3.} Comparar el rendimiento de las aplicaciones contenerizadas en diferentes entornos y arquitecturas, identificando las ventajas y desventajas de cada enfoque.
   \item \textbf{OB4.} Analizar los resultados obtenidos en los experimentos para identificar las limitaciones y desafíos asociados al uso de contenedores en entornos HPC, así como establecer futuras líneas de investigación.
\end{itemize}

