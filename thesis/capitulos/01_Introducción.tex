\chapter{Introducción}\label{cap:introduccion}

% [La introducción tiene que poner en contexto al lector contando, a modo de historia, el origen y contexto del problema, motivando por qué es necesario abordarlo y finalizando con lo que se propone en el proyecto.]

% [También es importante que sigas cierto orden y estructura a la hora de presentar (introducir) los contenidos, siguiendo un patrón que atienda al qué, para describir el contexto; al por qué, para dar razón o motivar el trabajo, y al por tanto, para definir objetivos consecuentes con la motivación y el contexto del trabajo.]

\section{Motivación}\label{sec:motivacion}
% [Opcional si se ha motivado la realización del proyecto en los párrafos anteriores.]

\section{Objetivos}\label{sec:objetivos}

Analizar la viabilidad y las limitaciones del uso de contenedores, concretamente Docker, para encapsular y ejecutar aplicaciones de alto rendimiento (HPC) en arquitecturas heterogéneas modernas —como big.LITTLE— y entornos multiplataforma, con el fin de facilitar su portabilidad, uso y adopción por parte de la comunidad científica.

\subsection{Objetivos específicos}\label{subsec:objetivos_especificos}

\begin{itemize}
   \item \textbf{OB1.} Investigar el estado del arte en el ámbito de la tecnología de contenedores y su aplicación en entornos de computación de alto rendimiento.
   \item \textbf{OB2.} Diseñar e implementar un conjunto de experimentos para evaluar el rendimiento de aplicaciones HPC contenerizadas en diferentes arquitecturas y plataformas.
   \item \textbf{OB3.} Comparar el rendimiento de las aplicaciones contenerizadas con su ejecución nativa en diferentes entornos y arquitecturas, identificando las ventajas y desventajas de cada enfoque.
   \item \textbf{OB4.} Analizar los resultados obtenidos en los experimentos para identificar las limitaciones y desafíos asociados al uso de contenedores en entornos HPC, así como establecer futuras líneas de investigación.
\end{itemize}

