\chapter{Gestión del Proyecto}\label{cap:planif}

\section{Tareas}

Tareas del OB1 (X):

\begin{itemize}
    \item 1. X
    \item 2. Y
\end{itemize}

\section{Planificación temporal}

\begin{table}[h!]
    \centering
    \begin{tabular}{|l|r|}
        \hline
        \textbf{Tarea} & \textbf{Horas estimadas} \\
        \hline
        Tarea 1        & 20                       \\
        Tarea 2        & 15                       \\
        Tarea 3        & 35                       \\
        Tarea 4        & 20                       \\
        Tarea 5        & 10                       \\
        Tarea 6        & 5                        \\
        \hline
        \textbf{Total} & \textbf{105}             \\
        \hline
    \end{tabular}
    \caption{Planificación temporal de tareas y horas estimadas}
    \label{tab:planificacion-temporal}
\end{table}

\section{Estimación de costes}

Los recursos necesarios para llevar a cabo el proyecto son:


\textbf{Hardware}

\begin{itemize}
    \item Ordenador portátil LG Gram 14Z90Q-G.AA75B, este equipo se utilizará para el desarrollo general del trabajo: creación del código para las pruebas, gestión de las pruebas en el clustery creación de la memoria. Cuenta con un procesador Intel Core i7-1260P, 16 GB de RAM y 512 GB de almacenamiento SSD.

    \item Ordenador portátil Lenovo Legion 5, será el equipo donde se ejecutarán las pruebas {TODO}. Cuenta con un procesador AMD Ryzen 7 4800H, 16 GB de RAM, 512 GB de almacenamiento SSD y una tarjeta gráfica NVIDIA RTX 2060 con 6 GB de VRAM.

    \item Ordenador portátil Apple MacBook Air M4, será el equipo de pruebas en entornos Apple. Cuenta con un procesador Apple M4 de {TODO}.

    \item Cluster de computación con 4 nodos, cada uno con {TODO}.
\end{itemize}

Para la ejecución de las pruebas en un cluster de computación, no se ha podido contar en el que pone la Universidad de Granada a disposición de los estudiantes, por lo que se ha optado por hacer uso de un cluster de computación en la nube, concretamente el servicio de Digital Ocean \footnote{\url{https://www.digitalocean.com/}}. Este servicio permite crear y gestionar clusters de computación con diferentes configuraciones de hardware, lo que facilita la ejecución de aplicaciones HPC en entornos distribuidos. Además, ofrece $200\$$ de crédito inicial para estudiantes, lo que permite realizar pruebas y experimentos sin coste adicional.

\textbf{Software}

\begin{itemize}
    \item Sistema operativo Ubuntu 24.04 LTS. Será la distribución Linux principal con la que vamos a trabajar, tanto en forma nativa, así como en los contenedores y en el cluster de computación.

    \item Sistema operativo Microsoft Windows 11. Será la distribución con la que se ejecutarán las pruebas de compatibilidad y rendimiento en entornos Windows.

    \item Sistema operativo macOS {TODO}. Será la distribución con la que se ejecutarán las pruebas de compatibilidad y rendimiento en entornos Apple.
\end{itemize}

\textbf{Recursos humanos}.

\begin{table}[h!]
    \centering
    \begin{tabular}{|l|l|r|}
        \hline
        \textbf{Dispositivo}             & \textbf{Descripción}             & \textbf{Coste (€)} \\
        \hline
        LG Gram 14Z90Q-G.AA75B           & Portátil principal de desarrollo & 1\,200             \\
        Lenovo Legion 5                  & Portátil de pruebas              & 1\,000             \\
        Apple MacBook Air M4             & Portátil de pruebas Apple        & 1\,059             \\
        Cluster de computación (4 nodos) & Nube Digital Ocean               & 400                \\
        \hline
        \textbf{Total}                   &                                  & \textbf{2\,600}    \\
        \hline
    \end{tabular}
    \caption{Costes estimados de hardware para el proyecto}
    \label{tab:costes-hardware}
\end{table}

En cuanto al software, los sistemas operativos Microsoft Windows 11 y macOS {TODO} vienen incluidos en los dispositivos correspondientes, por lo que no se ha considerado un coste adicional. El sistema operativo Ubuntu 24.04 LTS es de código abierto y gratuito, por lo que tampoco se ha considerado un coste adicional.

\textbf{Recursos humanos}

En la tabla \ref{tab:recursos-humanos} se detalla el coste por hora, las horas estimadas y el coste total de los recursos humanos necesarios para llevar a cabo el proyecto.

\begin{table}[h!]
    \centering
    \begin{tabular}{|l|l|r|r|r|}
        \hline
        \textbf{Recurso}        & \textbf{Puesto}  & \textbf{/h} & \textbf{Horas} & \textbf{Total (€)} \\
        \hline
        Fernando Cuesta Bueno   & Desarrollador    & 30          & 100            & 3\,000             \\
        Juan José Escobar Pérez & Tutor/Supervisor & 40          & 50             & 2\,000             \\
        \hline
        \textbf{Total}          &                  &             &                & \textbf{5\,000}    \\
        \hline
    \end{tabular}
    \caption{Costes estimados de recursos humanos para el proyecto}
    \label{tab:recursos-humanos}
\end{table}

\textbf{Coste total del proyecto}

El coste total del proyecto se calcula sumando los costes de hardware, software y recursos humanos. En la tabla \ref{tab:coste-total} se detalla el coste total estimado.

\begin{table}[h!]
    \centering
    \begin{tabular}{|l|r|}
        \hline
        \textbf{Concepto} & \textbf{Coste (€)} \\
        \hline
        Hardware          & 2\,600             \\
        Software          & 0                  \\
        Recursos humanos  & 5\,000             \\
        \hline
        \textbf{Total}    & \textbf{7\,600}    \\
        \hline
    \end{tabular}
    \caption{Coste total estimado del proyecto}
    \label{tab:coste-total}
\end{table}