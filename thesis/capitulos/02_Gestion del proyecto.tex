\chapter{Gestión del Proyecto}\label{cap:gestion_proyecto}

\section{Tareas}\label{sec:tareas}

\subsubsection{Tareas de organización del proyecto}\label{subsubsec:tareas_organizacion}
\begin{itemize}
    \item \textbf{Definición de objetivos y alcance del proyecto} \\
          Establecer claramente los objetivos principales y específicos del proyecto, así como el alcance y las limitaciones.

    \item \textbf{Planificación temporal} \\
          Desarrollar un cronograma detallado que incluya todas las fases del proyecto, desde la investigación inicial hasta la redacción final del informe.

    \item \textbf{Asignación de recursos} \\
          Identificar y asignar los recursos necesarios, tanto humanos como materiales, para llevar a cabo el proyecto de manera eficiente.

    \item \textbf{Gestión de riesgos} \\
          Identificar posibles riesgos que puedan afectar al desarrollo del proyecto y establecer planes de contingencia para mitigarlos.

    \item \textbf{Comunicación y seguimiento} \\
          Establecer canales de comunicación efectivos entre todos los miembros del equipo y realizar un seguimiento regular del progreso del proyecto para asegurar que se cumplen los plazos establecidos.
\end{itemize}

\subsubsection{Tareas del OB1 (Estudio del estado del arte)}\label{subsubsec:tareas_ob1}

\begin{itemize}
    \item \textbf{Revisión del estado del arte en HPC} \\
          Estudiar la evolución histórica y tendencias actuales en la computación de alto rendimiento.

    \item \textbf{Análisis del uso de contenedores en HPC} \\
          Revisar tecnologías de contenedores aplicadas a entornos científicos y de alto rendimiento. Comparar contenedores frente a máquinas virtuales en cuanto a eficiencia, overhead y portabilidad en HPC.

    \item \textbf{Estudio del uso de GPU en aplicaciones HPC} \\
          Revisar el papel de las GPUs en la aceleración de aplicaciones científicas y de ingeniería. Identificar librerías y frameworks para programación en GPU. Analizar casos de éxito en la integración de GPU en entornos HPC.

    \item \textbf{Investigación sobre el soporte de GPU en contenedores} \\
          Revisar soluciones actuales para ejecutar GPUs dentro de contenedores. Analizar el grado de compatibilidad con diferentes sistemas operativos y arquitecturas. Estudiar el impacto en rendimiento del uso de GPUs en entornos contenerizados en comparación con la ejecución nativa.
\end{itemize}

\subsubsection{Tareas del OB2 (Diseño e implementación de la propuesta)}\label{subsubsec:tareas_ob2}

\begin{itemize}
    \item \textbf{Selección de la aplicación o problema HPC a estudiar} \\
          Se seleccionará una aplicación representativa del ámbito HPC, justificando su elección en función de su relevancia científica, disponibilidad de código abierto y viabilidad técnica para su ejecución en diferentes plataformas y entornos contenerizados.

    \item \textbf{Preparar entornos y dependencias} \\
          Se identificarán y documentarán las librerías y herramientas necesarias, incluyendo MPI, OpenMP y CUDA. Se garantizará la homogeneidad de las configuraciones en todos los sistemas de prueba y se detallarán los requisitos específicos para cada plataforma (Linux, Windows, macOS).

    \item \textbf{Diseñar y construir imágenes de contenedor} \\
          Se desarrollarán Dockerfiles reproducibles que incluyan todas las dependencias necesarias, asegurando soporte para GPU mediante NVIDIA Container Toolkit. Las imágenes serán versionadas y almacenadas en un registro para facilitar su reutilización y trazabilidad.

    \item \textbf{Definir casos de prueba y parámetros de ejecución} \\
          Se establecerán experimentos mononodo variando el número de hebras, experimentos multinodo con diferentes cantidades de nodos y casos combinados que exploren el espacio de parámetros hebras × nodos.

    \item \textbf{Automatización y orquestación} \\
          Se implementarán scripts en para automatizar la ejecución de lotes de pruebas, así como la recogida y almacenamiento de logs y resultados.

    \item \textbf{Instrumentación y métricas} \\
          Se instrumentará la aplicación para medir tiempos totales de ejecución, uso de CPU y otros recursos. Se calcularán métricas como aceleración, eficiencia, throughput y overhead comparando la ejecución en contenedor frente a la nativa. Se generarán gráficos comparativos para el análisis de resultados.

    \item \textbf{Reproducibilidad y trazabilidad} \\
          Se mantendrá un repositorio con los Dockerfiles, scripts y documentación del proyecto. Se etiquetarán las versiones de las imágenes y dependencias para asegurar la reproducibilidad de los experimentos.
\end{itemize}

\subsubsection{Tareas del OB3 (Evaluación de rendimiento)}\label{subsubsec:tareas_ob3}

\begin{itemize}
    \item \textbf{Definición de criterios de comparación} \\
          Se establecerán las métricas principales para la comparación de rendimiento y se garantizará la paridad de configuraciones (versiones de compiladores, librerías, drivers).

    \item \textbf{Ejecución de pruebas comparativas} \\
          Se ejecutarán las mismas baterías de experimentos tanto en modo nativo como en contenedor. Se registrarán logs completos de cada ejecución.

    \item \textbf{Recopilación y organización de resultados} \\
          Se guardarán los tiempos de ejecución y métricas de uso de recursos, clasificando los datos según plataforma (Linux, Windows, macOS) y tipo de acelerador (CPU, GPU). Se establecerá un formato homogéneo para los ficheros de resultados (CSV o base de datos).

    \item \textbf{Análisis cuantitativo del rendimiento} \\
          Se calcularán diferencias absolutas y relativas entre ejecución nativa y contenerizada, estimando overheads medios y por caso. Se evaluará la escalabilidad en cada escenario y se aplicará análisis estadístico para validar la significancia de las diferencias.

    \item \textbf{Análisis cualitativo} \\
          Se identificarán ventajas no estrictamente de rendimiento (portabilidad, reproducibilidad, facilidad de despliegue) y se documentarán limitaciones observadas (drivers de GPU, gestión de red en contenedores, problemas de compatibilidad).

    \item \textbf{Visualización de resultados comparativos} \\
          Se generarán gráficos y tablas que destaquen los casos extremos (mejores y peores comportamientos), facilitando la interpretación de los resultados.
\end{itemize}

\subsubsection{Tareas del OB4 (Análisis de resultados)}\label{subsubsec:tareas_ob4}

\begin{itemize}
    \item \textbf{Revisión sistemática de los resultados experimentales} \\
          Se analizarán de manera estructurada los datos obtenidos en las pruebas, comparando el rendimiento entre ejecución nativa y contenerizada en las distintas plataformas (Linux, Windows, macOS) y ante el uso o no de aceleradores (CPU, GPU). Se identificarán tendencias generales, anomalías y comportamientos consistentes.

    \item \textbf{Detección de desafíos en la adopción de contenedores en HPC} \\
          Se evaluará la complejidad asociada a la configuración, despliegue y mantenimiento de entornos contenerizados en HPC, incluyendo la integración de aceleradores como GPU y la gestión de dependencias específicas.

    \item \textbf{Propuesta de líneas de investigación futura} \\
          A partir de los resultados y desafíos identificados, se propondrán posibles líneas de trabajo futuro.
\end{itemize}

\section{Planificación temporal}

En la tabla \ref{tab:planificacion-temporal} se presenta una estimación del tiempo necesario para completar cada una de las tareas principales del proyecto, desglosado en horas dedicadas por el desarrollador y el tutor.

\begin{table}[!ht]
    \centering
    \setlength{\tabcolsep}{3pt}
    \renewcommand{\arraystretch}{1.1}
    \begin{tabular}{|p{2.5cm}|r|r|}
        \hline
        \textbf{Tarea}  & \textbf{Desarrollador (h)} & \textbf{Tutor (h)} \\
        \hline
        Planificación   & 20                         & 5                  \\
        Estado del arte & 40                         & 10                 \\
        Implementación  & 85                         & 8                  \\
        Evaluación      & 55                         & 7                  \\
        Análisis        & 30                         & 5                  \\
        \hline
        \textbf{Total}  & \textbf{230}               & \textbf{35}        \\
        \hline
    \end{tabular}
    \caption{Planificación temporal de tareas y horas estimadas}
    \label{tab:planificacion-temporal}
\end{table}



\section{Estimación de costes}

Los recursos necesarios para llevar a cabo el proyecto son:


\textbf{Hardware}

\begin{itemize}
    \item Ordenador portátil LG Gram 14Z90Q-G.AA75B, este equipo se utilizará para el desarrollo general del trabajo: creación del código para las pruebas, gestión de las pruebas en el clustery creación de la memoria. Cuenta con un procesador Intel Core i7-1260P, 16 GB de RAM y 512 GB de almacenamiento SSD.

    \item Ordenador portátil Lenovo Legion 5, será el equipo donde se ejecutarán las pruebas {TODO}. Cuenta con un procesador AMD Ryzen 7 4800H, 16 GB de RAM, 512 GB de almacenamiento SSD y una tarjeta gráfica NVIDIA RTX 2060 con 6 GB de VRAM.

    \item Ordenador portátil Apple MacBook Air M4, será el equipo de pruebas en entornos Apple. Cuenta con un procesador Apple M4 de {TODO}.

    \item Cluster de computación con 4 nodos, cada uno con {TODO}.
\end{itemize}

Para la ejecución de las pruebas en un cluster de computación, no se ha podido contar en el que pone la Universidad de Granada a disposición de los estudiantes, por lo que se ha optado por hacer uso de un cluster de computación en la nube, concretamente el servicio de Digital Ocean \footnote{\url{https://www.digitalocean.com/}}. Este servicio permite crear y gestionar clusters de computación con diferentes configuraciones de hardware, lo que facilita la ejecución de aplicaciones HPC en entornos distribuidos. Además, ofrece $200\$$ de crédito inicial para estudiantes, lo que permite realizar pruebas y experimentos sin coste adicional.

\textbf{Software}

\begin{itemize}
    \item Sistema operativo Ubuntu 24.04 LTS. Será la distribución Linux principal con la que vamos a trabajar, tanto en forma nativa, así como en los contenedores y en el cluster de computación.

    \item Sistema operativo Microsoft Windows 11. Será la distribución con la que se ejecutarán las pruebas de compatibilidad y rendimiento en entornos Windows.

    \item Sistema operativo macOS {TODO}. Será la distribución con la que se ejecutarán las pruebas de compatibilidad y rendimiento en entornos Apple.
\end{itemize}

\textbf{Recursos humanos}.

\begin{table}[!ht]
    \centering
    \begin{tabular}{|l|l|r|}
        \hline
        \textbf{Dispositivo}             & \textbf{Descripción}             & \textbf{Coste (€)} \\
        \hline
        LG Gram 14Z90Q-G.AA75B           & Portátil principal de desarrollo & 1\,200             \\
        Lenovo Legion 5                  & Portátil de pruebas              & 1\,000             \\
        Apple MacBook Air M4             & Portátil de pruebas Apple        & 1\,059             \\
        Cluster de computación (4 nodos) & Nube Digital Ocean               & 400                \\
        \hline
        \textbf{Total}                   &                                  & \textbf{2\,600}    \\
        \hline
    \end{tabular}
    \caption{Costes estimados de hardware para el proyecto}
    \label{tab:costes-hardware}
\end{table}

En cuanto al software, los sistemas operativos Microsoft Windows 11 y macOS {TODO} vienen incluidos en los dispositivos correspondientes, por lo que no se ha considerado un coste adicional. El sistema operativo Ubuntu 24.04 LTS es de código abierto y gratuito, por lo que tampoco se ha considerado un coste adicional.

\textbf{Recursos humanos}

En la tabla \ref{tab:recursos-humanos} se detalla el coste por hora, las horas estimadas y el coste total de los recursos humanos necesarios para llevar a cabo el proyecto.

\begin{table}[!ht]
    \centering
    \begin{tabular}{|l|l|r|r|r|}
        \hline
        \textbf{Recurso}        & \textbf{Puesto}  & \textbf{/h} & \textbf{Horas} & \textbf{Total (€)} \\
        \hline
        Fernando Cuesta Bueno   & Desarrollador    & 30          & 100            & 3\,000             \\
        Juan José Escobar Pérez & Tutor/Supervisor & 40          & 50             & 2\,000             \\
        \hline
        \textbf{Total}          &                  &             &                & \textbf{5\,000}    \\
        \hline
    \end{tabular}
    \caption{Costes estimados de recursos humanos para el proyecto}
    \label{tab:recursos-humanos}
\end{table}

\textbf{Coste total del proyecto}

El coste total del proyecto se calcula sumando los costes de hardware, software y recursos humanos. En la tabla \ref{tab:coste-total} se detalla el coste total estimado.

\begin{table}[!ht]
    \centering
    \begin{tabular}{|l|r|}
        \hline
        \textbf{Concepto} & \textbf{Coste (€)} \\
        \hline
        Hardware          & 2\,600             \\
        Software          & 0                  \\
        Recursos humanos  & 5\,000             \\
        \hline
        \textbf{Total}    & \textbf{7\,600}    \\
        \hline
    \end{tabular}
    \caption{Coste total estimado del proyecto}
    \label{tab:coste-total}
\end{table}